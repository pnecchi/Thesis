\chapter{Asset Allocation}

The asset allocation problem consists of determining how to dynamically invest
the available capital in a portfolio of different assets in order to maximize
the expected total return or another relevant performance measure. Let us
consider a financial market consisting of $I+1$ different stocks that are
traded only at discrete times $t \in \{0, 1, 2, \ldots\}$ and denote by $Z_t =
{(Z_t^0, Z_t^1, \ldots, Z_t^I)}^T$ their prices at time t. Typically, $Z_t^0$
refers to a riskless asset whose dynamic is given by $Z_t^0 = {(1 + R)}^t$ and
$R$ is the deterministic risk-free interest rate. The investment process works
as follows: at time $t$, the investor observes the state of the market $S_t$,
consisting for example of the past asset prices and other relevant economic
variables, and subsequently chooses how to rebalance his portfolio, by
specifying the units of each stock $n_t = {(n_t^0 , n_t^1 , \ldots , n_t^I)}^T$
to be held between $t$ and $t+1$. In doing so, he needs to take into account
the transaction costs that he has to pay to the broker to change his position.
At time $t+1$, the investor realizes a profit or a loss from his investment due
to the stochastic variation of the stock values. The investor’s goal is to
maximize a given performance measure.

\section{Reward Function}
Let $W_t$ denote the wealth of the investor at time $t$. The profit realized
between $t$ and $t+1$ is simply given by the difference between the trading
P\&L and the transaction costs payed to the broker, which we assume
proportional to the value of the transaction. More formally
\begin{equation*}
	\Delta W_{t+1} = W_{t+1} - W_t = \sum^{I}_{i=0} \left[n_t^m (Z_{t+1}^i -
	Z_t^i) - \delta_i \left| n_t^i - n_{t-1}^i \right| Z_t^i \right] 
\end{equation*}
The portfolio return between $t$ and $t+1$ is thus given by
\begin{equation}\label{eq:portfolio_return}
	X_{t+1} = \frac{\Delta W_{t+1}}{W_t} = \sum^{I}_{i=0} \left[ a_t^i
	X_{t+1}^i - \delta_i \left| a_t^i - \tilde{a}_t^i \right| \right]  
\end{equation}
where 
\begin{equation*}
	X_{t+1}^i = \frac{\Delta Z_{t+1}^i}{Z_t^i}
\end{equation*}
is the return of the $i$-th stock between $t$ and $t+1$, 
\begin{equation*}
	a_t^i = \frac{n_t^i Z_t^i}{W_t}
\end{equation*}
is the fraction of wealth invested in the $i$-th stock between time $t$ and
$t+1$ and finally 
\begin{equation*}
	\tilde{a}_t^i = \frac{n_{t-1}^i Z_t^i}{W_t} = \frac{a_{t-1}^i (1+X_t^i)}
	{1 + X_t}
\end{equation*}
is the fraction of wealth invested in the $i$-th stock just before the 
reallocation. We assume that the agent invests all his wealth at each step, so 
that $W_t$ can be also interpreted as the value of his portfolio. This 
assumption leads to the following constraint on the portfolio weights
\begin{equation}
	\sum^{I}_{i=0} a_t^i = 1 \;\;\;\;\; \forall t \in \{0, 1, 2, \ldots\}
\end{equation}
Plugging this constraint into Eq. (\ref{eq:portfolio_return}), we obtain
\begin{equation}\label{eq:portfolio_return_benchmark}
	X_{t+1} = R + \sum^{I}_{i=1} a_t^i (X_{t+1}^i - R) - \sum^{I}_{i=0}
	\delta_i \left| a_t^i - \tilde{a}_t^i \right|  
\end{equation}
which highlights the role of the risk-free asset as a benchmark for the 
portfolio returns. The total profit realized by the investor between $t=0$ and
$T$ is 
\begin{equation*}
	P_{0,T} = W_T - W_0 = \sum^{T}_{t=1} \Delta W_t = \sum^{T}_{t=1} W_t R_t  
\end{equation*}
The portfolio return between $t=0$ and $T$ is given by
\begin{equation}
	X_{0:T} = \frac{W_T}{W_0} = \prod_{t=1}^T (1+X_t) - 1
\end{equation}
In order to cast the asset allocation problem in the reinforcement learning
framework, we consider the log-return of the portfolio between $t=0$ and $T$
\begin{equation}
	r_{0,T} = \log \frac{W_T}{W_0} = \sum^{T}_{t=1} \log(1+X_t) = \sum_{t=1}^T
	r_t
\end{equation}
where $r_{t+1}$ is the log-return of the portfolio between $t$ and $t+1$
\begin{equation}
	r_{t+1} \log \left\{ 1 + \sum^{I}_{i=0} \left[ a_t^i X_{t+1}^i - \delta_i
	\left| a_t^i - \tilde{a}_{t-1}^i \right| \right] \right\}
\end{equation}
The portfolio return and log-return can be used as the reward function of a
RL algorithm, either in a offline or in an online approach. In the case where
only two assets are available on the market - a risky assez and a risk-free
asset - and that no transaction costs are payed on the risk-free asset, the
returns take the simpler form
\begin{equation}
	X_{t+1} = R + a_t (X_{t+1}^1 - R) - \delta \left| a_t - \tilde{a}_t \right|
\end{equation} 

\section{States}

\section{Actions}
In this section, we specify the actions that the trading system may take. If we
assume that the agent invests all his capital at each time step and that
short-selling is not allowed, then the portfolio weights should satisfy
\begin{equation} 
	a_t^i \geq 0 \;\;\;\;\;\; \sum^{I}_{i=0} a_t^i = 1
\end{equation} 
This constraint can be easily enforced by considering a softmax policy. If
short selling is allowed, weights might also be negative. However, working in a
constinuous action space becomes computationally difficult.  Therefore, in the
two assets scenario, we will consider the simplified problem where $a_t \in
\{-1, 0, +1\}$. Thus the agent may be long ($+1$), neutral ($0$) or short
($-1$) on the risky-asset. Working in a discrete action space is more simple,
and standard value-based approaches might be employed. We could generalize the
framework of the previous sections to include additional fees or margin
requirements if short positions are taken by the agent.

