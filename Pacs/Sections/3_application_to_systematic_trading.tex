\section{Application to Systematic Trading}
\label{sec:application_to_systematic_trading}

The asset allocation problem consists of determining how to dynamically invest
the available capital in a portfolio of different assets in order to maximize
the expected total return or another relevant performance measure. Let us
consider a financial market consisting of $I+1$ different stocks that are
traded only at discrete times $t \in \{0, 1, 2, \ldots\}$ and denote by
${Z}_t = {(Z_t^0, Z_t^1, \ldots, Z_t^I)}^T$ their prices at time $t$.
Typically, $Z_t^0$ refers to a riskless asset whose dynamic is given by $Z_t^0
= {(1 + X)}^t$ where $X$ is the deterministic risk-free interest rate. The
investment process works as follows: at time $t$, the investor observes the
state of the market $S_t$, consisting for example of the past asset prices and
other relevant economic variables, and subsequently chooses how to rebalance
his portfolio, by specifying the units of each stock ${n}_t = {(n_t^0 ,
n_t^1 , \ldots , n_t^I)}^T$ to be held between $t$ and $t+1$. In doing so, he
needs to take into account the transaction costs that he has to pay to the
broker to change his position.  At time $t+1$, the investor realizes a profit
or a loss from his investment due to the stochastic variation of the stock
values. The investor’s goal is to maximize a given performance measure.