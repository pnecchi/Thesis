\documentclass[a4paper,11pt]{article}

\usepackage{Style/thesis}

\begin{document}

\title{Draft: Online PGPE}
\author{Pierpaolo Necchi}
\maketitle

In PGPE we consider a deterministic controller $F: \S \times \Theta \to \A$
that, given a set of parameters $\theta \in \Theta \subseteq \R^{D_\theta}$,
maps a state $s \in \S$ to an action $a = F(s; \theta) = F_\theta(s) \in \A$. 
The policy parameters are drawn from a probability distribution $p_\xi$, with
hyperparameters $\xi \in \Xi \subseteq \R^{D_\xi}$. Combining these two
hypotheses, the agent follows a stochastic policy $\pi_\xi$ given by
\begin{equation}
	\pi_\xi(s,a) = \pi(s, a; \xi) =	\int_\Theta p_\xi(\theta)
	\ind{F_{\theta}(s)=a} d\theta
\end{equation}
If we assume that the policy $\pi_\xi$ is differentiable with respect to $\xi$,
the policy gradient theorem states that the gradient of the objective function
is given by 
\begin{equation}\label{eq:policy_gradient}
	\begin{split}
		\nabla_\xi J(\xi) &= \E{\nabla_\xi \log \pi_\xi(S,A) Q_{\pi_\xi}(S,A)}\\
						  &= \int_{\S} \mu_{\xi}(s) \int_{\A} \pi_{\xi}(s, a)
		\nabla_{\xi} \log \pi_{\xi}(s,a) Q_{\pi_\xi}(s,a) da ds\\
	\end{split}
\end{equation}
This expression can be rewritten in a different form that will allow us to
formulate various online versions of the PGPE algorithm, similar to the
standard policy gradient algorithms. First, by using the likelihood trick we
have
\begin{equation*}
	\begin{split}
		\pi_\xi(s,a) \nabla_\xi \log \pi_\xi(s,a) &= \nabla_\xi \pi_\xi(s,a) \\
			&= \int_\Theta \nabla_\xi p_\xi(\theta) \ind{F_\theta(s)=a} d\theta
		\\
		&= \int_\Theta p_\xi(\theta) \nabla_\xi \log p_\xi(\theta) \ind{F_\theta(s)=a} d\theta
		\\
		&= \E[\theta \sim p_\xi]{\nabla_\xi \log p_\xi(\theta) \ind{F_\theta(s)=a}}
	\end{split}
\end{equation*}
Then, by exchanging the integral over the action space with the expectation
\begin{equation*}
	\begin{split}
		\int_\A \pi_\xi(s,a) \nabla_\xi \log \pi_\xi(s,a) Q_{\pi_\xi}(s,a) da 
			&= \E[\theta \sim p_\xi]{\nabla_\xi \log p_\xi(\theta) \int_\A \ind{F_\theta(s)=a}
			Q_{\pi_\xi}(s,a) da}\\
			&= \E[\theta \sim p_\xi]{\nabla_\xi \log p_\xi(\theta) Q_{\pi_\xi}(s, F_\theta(s))}\\
			&= \E[\theta \sim p_\xi]{\nabla_\xi \log p_\xi(\theta) Q_{\pi_\xi}(s, \theta))}
	\end{split}
\end{equation*}
Finally, pluggin this equality in the policy gradient theorem, we obtain 
\begin{equation}
	\begin{split}
		\nabla_\xi J(\xi) &= \E{\nabla_\xi \log p_\xi(\theta) Q_{\pi_\xi}(S,
	\theta)}\\
		&= \int_\S \mu_\xi(s) \int_\Theta p_\xi(\theta) \nabla_\xi \log
	p_\xi(\theta) Q_{\pi_\xi}(S, \theta) d\theta ds
	\end{split}
\end{equation}
This expression is very similar to the original policy gradient theorem, but
the expectation is taken over the controller parameters and not on the actions.
Thus, we might interpret this result as if the agent chose the parameters
$\theta$, which then lead to an action through the deterministic mapping
$F_\theta$. The agent policy is in parameters space and not in the control
space. As in the standard policy gradient methods, we can add a state-dependent
baseline $B(S)$ to the gradient without increasing the bias
\begin{equation}
	\nabla_\xi J(\xi) = \E{\nabla_\xi \log p_\xi(\theta) \left(Q_{\pi_\xi}(S,
						\theta) - B(S)\right)}
\end{equation}
Indeed, 
\begin{equation*}
	\begin{split}
		\E{\nabla_\xi \log p_\xi(\theta) B(S)} 
			&= \int_\S \mu_\xi(s) \int_\Theta p_\xi(\theta) \nabla_\xi \log
		p_\xi(\theta) B(s) d\theta ds\\
		&= \int_\S \mu_\xi(s) B(s) ds \underbrace{\int_\Theta \nabla_\xi p_\xi(\theta)
	d\theta}_{\nabla_\xi 1 = 0} = 0
	\end{split}
\end{equation*}
This result can be used to design several actor-only or actor-critic online
algorithms that are the parameter-based equivalents of the traditional 
control-based policy-gradient algorithms, such as REINFORCE, GPOMDP or QAC.  

\end{document}
